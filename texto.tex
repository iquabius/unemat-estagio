\documentclass[relatorio]{tex/unemat-comp}
% Opções da classe inf-ufg (ao usar mais de uma, separe por vírgulas)
%   [tese]         -> Tese de doutorado.
%   [dissertacao]  -> Dissertação de mestrado (padrão).
%   [monografia]   -> Monografia de especialização.
%   [relatorio]    -> Relatório final de graduação.
%   [abnt]         -> Usa o estilo "abnt-alf" de citação bibliográfica.
%   [nocolorlinks] -> Os links de navegação no texto ficam na cor preta.
%                     Use esta opção para gerar o arquivo para impressão
%                     da versão final do seu texto!!!

% This package is useful for adding todo notes to the side of the text.
% Use 'disable' option to remove all todo notes from output.
% Run 'texdoc todonotes' in the terminal for help.
\usepackage[textsize=scriptsize]{todonotes}
\setlength{\marginparwidth}{3cm}
\reversemarginpar{}

% \addbibresource{refs.bib}

%----------------------------------------------------- INICIO DO DOCUMENTO %
\begin{document}

%------------------------------------------ AUTOR, TÍTULO E DATA DE DEFESA %
\autor{Josias Duarte Busiquia}
\autorR{Busiquia, Josias Duarte}

% \titulo{Interfaces Gráficas Declarativas}
% \subtitulo{Demonstração e Análise de Programação Funcional e Reativa}
\titulo{Website para as Olimpíadas de Matemática da UNEMAT}

\cidade{Barra do Bugres--MT}
% Formato numérico: \dia{01}, \mes{01} e \ano{2009}
\dia{20}
\mes{02}
\ano{2020}

%-------------------------------------------------------------- ORIENTADOR %
\orientador{Me. Alexandre Berndt}
\orientadorR{Berndt, Alexandre}


%-------------------------------------------------- INSTITUIÇÃO E PROGRAMA %
\universidade{Universidade do Estado de Mato Grosso}
\uni{UNEMAT}
\unidade{Faculdade de Ciências Exatas e Tecnológicas}
\departamento{Departamento de Ciência da Computação} % Unidades com mais de um departamento.

\programa{Ciência da Computação}
\concentracao{Ciência da Computação}

%-------------------------------------------------- ELEMENTOS PRÉ-TEXTUAIS %
\capa{}    % Gera o modelo da capa externa do trabalho
\rosto{}   % Primeira folha interna do trabalho
% TODO: Colocar página de dados do estágio

\tabelas[nada]

%Opções:
%nada [] -> Gera apenas o sumário
%fig     -> Gera o sumário e a lista de figuras
%tab     -> Sumário e lista de tabelas
%alg     -> Sumário e lista de algoritmos
%cod     -> Sumário e lista de códigos de programas
%
% Pode-se usar qualquer combinação dessas opções.
% Por exemplo:
%  figtab       -> Sumário e listas de figuras e tabelas
%  figtabcod    -> Sumário e listas de figuras, tabelas e
%                  códigos de programas
%  figtabalg    -> Sumário e listas de figuras, tabelas e algoritmos
%  figtabalgcod -> Sumário e listas de figuras, tabelas, algoritmos e
%                  códigos de programas

%--------------------------------------------------------------- CAPÍTULOS %
\chapter{Introdução}\label{chap:1-intro}
Parte principal do relatório. Em dois parágrafos apresente de maneira sucinta o
\emph{\textbf{assunto}} tratado, aponte a \emph{\textbf{relevância}} do trabalho e o \emph{\textbf{local}} onde o
estágio foi realizado.

No terceiro parágrafo, \emph{\textbf{justifique}} a motivação para a escolha do tema/área e
das tecnologias e ferramentas usadas.

No último parágrafo descreva resumidamente as principais atividades
desenvolvidas.


\chapter{Objetivos}\label{chap:2-objetivos}
\section{Objetivo Geral}
\label{sec:org872d0bb}

Um dos objetivos do programa de extensão das Olimpíadas de Matemática é o
“desenvolvimento de um sistema computacional que contemple todas as etapas de
inscrição, acompanhamento, correção, gestão e tratamento estatístico dos
dados”.

\section{Objetivos Específicos}
\label{sec:orgd7fe0f9}

Faz sentido listar os módulos da aplicação como objetivos específicos?

\begin{itemize}
\item Contemplar criação e acompanhamento de edições das olimpíadas;
\item Contemplar criação e gerenciamento de questões;
\item Contemplar elaboração e correção de provas;
\item Contemplar inscrição e acompanhamento das escolas;
\item Contemplar inscrição e acompanhamento dos alunos;
\item Contemplar tratamento estatístico dos dados;
\end{itemize}


\chapter{Atividades Desenvolvidas}\label{chap:3-atividades}
\input{texto/3-atividades}

\chapter{Sugestões e Recomendações}\label{chap:4-sugestões}
Aponte as necessidades de melhoria identificadas na empresa/organização onde o
estágio foi desenvolvido. Apresente novas possibilidades de estudo e/ou
desenvolvimento de estágio no tema desenvolvido.


\chapter{Considerações Finais}\label{chap:5-consideracoes}
\input{texto/5-consideracoes}


%------------------------------------------------------------- REFERÊNCIAS %

\cleardoublepage{}

%\nocite{*}                  % para incluir referências não citadas no texto

\arial{}
\printbibliography[heading=bibintoc,title=\refname]


%------------------------------------------------------------------------- %
%               F I M   D O  A R Q U I V O :  t e x t o . t e x            %
%------------------------------------------------------------------------- %
\end{document}
%%% Local Variables:
%%% mode: latex
%%% TeX-master: t
%%% End:
